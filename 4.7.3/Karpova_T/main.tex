\documentclass[a4paper]{article}
\usepackage[utf8]{inputenc}
\usepackage[russian]{babel}
\usepackage[T2]{fontenc}
\usepackage[warn]{mathtext}
\usepackage{graphicx}
\usepackage{amsmath}
\usepackage{floatflt}
\usepackage[left=20mm, top=20mm, right=20mm, bottom=20mm, footskip=10mm]{geometry}


\graphicspath{ {images/} }
\usepackage{multicol}
\setlength{\columnsep}{2cm}


\begin{document}

\begin{titlepage}
	\centering
	\vspace{5cm}
	{\scshape\LARGE Московский физико-технический институт \par}
	\vspace{4cm}
	{\scshape\Large Лабораторная работа \par}
	\vspace{1cm}
	{\huge\bfseries Поляризация \par}
	\vspace{1cm}
	\vfill
\begin{flushright}
	{\large выполнили студент 653 группы ФФКЭ}\par
	\vspace{0.3cm}
	{\LARGE Давыдов Валентин} \par
		\vspace{0.3cm}
	{\LARGE }
\end{flushright}
	

	\vfill

% Bottom of the page
	Долгопрудный, 2018 г.
\end{titlepage}

\section{Цель работы:}
Ознакомление с методами получения и анализа поляризованного света

\section{В работе используются:}
\begin{itemize}
    \item оптическая скамья с осветителем
    \item зелёный светофильтр
    \item два поляроида
    \item чёрное зеркало
    \item полированная эбонитовая пластинка
    \item стопа стеклянных пластинок
    \item пластинки в $1/4$ и $1/2$ длины волны
    \item пластинка в одну длину волны для зелёного света (пластинка чувствительного оттенка)
\end{itemize}

\section{Теоретические положения}

При помощи специальных приспособлений (поляризаторов), естественный свет может быть превращен в линейно поляризованный (или, как иногда говорят, в плоскополяризованный). В линейно поляризованной световой волне пара векторов \textbf{E} и \textbf{H} не изменяет с течением времени своей ориентации. Плоскость \textbf{E, S} называется в этом случае \textit{плоскостью колебаний}. \par
Наиболее общим типом поляризации является \textit{эллиптическая поляризация}. В эллиптически поляризованной световой волне конец вектора
\textbf{E} (в данной точке пространства) описывает некоторый эллипс. Линейно
поляризованный свет можно рассматривать как частный случай эллиптически поляризованного света, когда эллипс поляризации вырождается в отрезок прямой линии; другим частным случаем является круговая
поляризация (эллипс поляризации является окружностью). \par
Для получения линейно поляризованного света применяются специальные оптические приспособления — поляризаторы. Направление колебаний электрического вектора в волне, прошедшей через поляризатор, называется
разрешенным направлением поляризатора.
Всякий поляризатор может быть использован для исследования поляризованного света, т. е. в качестве анализатора. Интенсивность I линейно поляризованного света после прохождения через анализатор за-
висит от угла, образованного плоскостью колебаний с разрешенным направлением анализатора:
\begin{equation}
  I = I_0 \cos 2\alpha.  
\end{equation}
Соотношение (1) носит название \textit{закона Малюса}. \par
. Отраженный от диэлектрика свет всегда частично поляризован. Степень поляризации света, отраженного от диэлектрической пластинки в воздух, зависит от показателя преломления диэлектрика $n$ и от угла падения $i$. Как следует из формул Френеля, полная поляризация отраженного света достигается
при падении под \textit{углом Брюстера}, который определяется соотношением
\begin{equation}
 \tg i = n.   
\end{equation}

В этом случае плоскость колебаний электрического вектора в отраженном свете перпендикулярна плоскости падения. Для увеличения степени поляризации преломлённого
света используют стопу стеклянных пластинок, расположенных под углом Брюстера к падающему свету. \par

\textbf{Определение направления разрешенной плоскости колебаний поляроида.} При падении свет
а на отражающую поверхность по
д углом Брюстера
свет в отражённом луче полностью поляризован,
а вектор
E параллелен
отражающей поверхности («правило иголки»). Луч света, прошедший поляроид и отразившийся от чёрного зеркала, имеет минимальную интенсивность при выполнении двух условий: во-первых, свет падает на отражающую поверхность под углом Брюстера, и во-вторых, в падающем пучке вектор
E лежит в плоскости падения. \par
Вращая поляроид вокруг направления луча
и чёрное зеркало вокруг оси, перпендикулярной лучу, методом последовательных приближений
можно добиться минимальной яркости луча, отражённого от зеркала, и таким образом определить разрешённое направление поляроида. Измеряя угол поворота зеркала (угол Брюстера), нетрудно определить коэффициент преломления материала, из которого изготовлено зеркало \par


\section{Ход работы}
\subsection{Определение разрешённых направлений поляроидов}
\begin{enumerate}
    \item Разместим на оптической скамье осветитель
S, поляроид P1 и чёрное зеркало (пластинку чёрного стекла) так, чтобы плоскость падения была горизонтальна. Свет, отражённый от зеркала, рассматриваем сбоку, расположив глаз таким образом, чтобы вблизи оси вращения зеркала можно было увидеть изображение диафрагмы осветителя.
Поворачивая поляроид вокруг направления луча, добьёмся наименьшей яркости отражённого пятна. Оставим поляроид в этом положении
и вращением зеркала вокруг вертикальной оси снова добьёмся минимальной интенсивности отражённого луча.
\par Для первого поляроида разрешённое направление горизонтальное, на лимбе $336^{\circ}$
\item Вместо чёрного зеркала поставим второй поляроид. Скрестим их, определим разрешённое направление второго поляроида - горизонтальная волна, на лимбе $40^{\circ}$
\end{enumerate}

\subsection{Определение угла Брюстера для эбонита}
\begin{enumerate}
\item Поставим на скамью вместо чёрного зеркала эбонитовую пластину с круговой шкалой.

\item  Повернем эбонитовое зеркало вокруг вертикальной оси так, чтобы его
плоскость была перпендикулярна лучу, и попытаемся совместить отражённое от эбонита пятно с отверстием осветителя.

\item Установите направление разрешённых
колебаний поляроида P1 горизонтально и найдите угол поворота эбонита
$\varphi_$Б, при котором интенсивность
отражённого луча минимальна: его абсолютное значение равно $304^{\circ}$

\item Повторите измерения, добавив светофильтр Ф, и сравним результаты - они получились одинаковыми

\item По углу Брюстера рассчитайте показатель преломления эбонита и сравните с табличным.
\begin{center}
    $n = \tg \varphi = \tg -304^{\circ} = 1.48$
\end{center}
Табличное значение показателя преломления эбонита $n = 1.6$

\end{enumerate}

\subsection{Исследование стопы}

Поставим стопу стеклянных пластинок вместо эбонитового зеркала
и подберем для неё такое положение, при
котором свет падает на стопу под углом Брюстера.
Осветим стопу неполяризованным светом (снимите поляризатор с оптической скамьи) и, рассматривая через поляроиды свет, отражённый от стопы, определим ориентацию вектора
\textbf{E} в отражённом луче; затем определим
характер поляризации света в преломлённом луче. 
\par Наблюдая прошедший через стопу стеклянных пластинок луч света, убеждаемся, в
том что плоскости поляризации у отраженного и преломленного лучей взаимно
перпендикулярны: Угол на лимбе $P_1 = 328^{\circ}$, $P_2 = 235^{\circ}$. Преломленные лучи горизонтальные, отраженные – вертикальные. Установили, что лучи имеют правый
круговой тип поляризации. 

\subsection{Определение главных плоскостей двоякопреломляющих пластин}
Поставим кристаллическую пластинку
между скрещенными поляроидами $P_1$ и $P_2$. Вращая пластинку вокруг направления луча
и наблюдая за интенсивностью света, про
ходящего сквозь второй поляроид, определите, при
каком условии главные направления пластинки
совпадают с разрешёнными направлениями поляроидов. Повторите опыт для второй пластинки.

\begin{center}
    Пластина 1 \hspace{1cm} Пластина 2 \\
    $min: 62^{\circ}, 152^{\circ}$ \hspace {1cm} $min: 18^{\circ}, 109^{\circ}$ \\
     $max: 106^{\circ}, 202^{\circ}$ \hspace {1cm} $max: 70^{\circ}, 160^{\circ}$
\end{center}
Минимумы и максимумы интенсивности чередуются через $45^{\circ}$, главные плоскости пластин совпадают с разрешёнными направлениями поляроидов при максимальной интенсивности

\subsection{Выделение пластин $\lambda/2$ и $\lambda/4$}
Добавим к схеме зелёный фильтр; установим
разрешённое направление поляроида горизонтально, а главные направления исследуемой пластинки — под углом $45^{\circ}$ к горизонтали. 
\par Пластинка $\lambda/2$ не меняет характер поляризации, при её повороте \textit{меняется интенсивность}, а поляризация остаётся линейной. 
\par Пластинка $\lambda/4$ создаёт сдвиг фаз $\pi/2$ между колебаниями - эллиптическая поляризация. Эта пластинка \textit{не меняет интенсивность} при повороте.

\subsection{Определение направлений большей и меньшей скоростей в пластинке $\lambda/4$}

\begin{enumerate}
    \item Поставим между скрещенными поляроидами пластинку чувствительного оттенка (λ для зелёного света), имеющую вид стрелки. Световой вектор, ориентированный вдоль направления стрелки, проходит с большей скоростью, перпендикулярный — с меньшей. \par
Установиv разрешённое направление первого поляроида горизонтально и убедимся с помощью второго поляроида, что эта пластинка
не меняет поляризацию зелёного света в условиях предыдущего опыта.

\item Уберем зелёный фильтр
и поставим между скрещенными поляроидами
пластинку $\lambda$ (стрелка под углом $45^{\circ}$ к разрешённым направлениям поляроидов).
Глядя сквозь второй поляроид на стрелку, убедимся, что она имеет пурпурный цвет (зелёный свет задерживается вторым поляроидом, а красная и синяя компоненты проходят).
\item . Добавим
к схеме пластинку $\lambda/4$, главные направления
которой совпадают с главными направлениями пластины $\lambda$ и ориентированы под
углом $45^{\circ}$ к разрешённым направлениям скрещенных поляроидов

\item Теперь уберём пластину чувствительного оттенка. После второго поляроида интенсивность минимальная - значит, быстрая ось пластинки направлена горизонтально, направление вращения правое, направление колебаний в первом и третьем квадрантах (разность фаз $\pi/4$). \par
При повороте рейтера со стрелкой на 180◦ вокруг вертикальной оси
цвет стрелки меняется от зелёно-голубого до оранжево-жёлтого.

\end{enumerate}

\subsection{Определение направления вращения светового вектора в
эллиптически поляризованной волне}

\begin{enumerate}
    \item  Снова поставим зелёный фильтр,
а за ним между скрещенными поляроидами
— пластинку произвольной толщины (
$\lambda/4$ ).
\item Получим эллиптически-поляризованный свет. Для этого установим разрешённое направление первого поляроида под углом 10–20◦ к горизонтали так, чтобы вектор \textbf{E} падающего на пластинку света был расположен в первом квадранте.
Установим разрешённое направление второго поляроида вертикально и, вращая пластинку, найдем минимальную
интенсивность света, прошедшего второй поляроид. Вращая второй поляроид, убедитесь, что свет поляризован эллиптически,
а не линейно.
Таким образом, получим эллипс поляризации с вертикально ориентированной малой осью.
\item  Для определения направления вращения светового вектора в эллипсе
установим между поляроидами дополнительную пластинку λ/4 с известными направлениями «быстрой» и «медленной» осей, ориентированными по осям эллипса поляризации анализируемого света.
В этом случае вектор \textbf{E} на выходе будет таким, как если бы свет прошёл две
пластинки λ/4: свет на выходе из второй пластинки будет линейно поляризован. Если пластинки поодиночке дают эллипсы, вращающиеся в разные стороны, то поставленные друг за другом, они скомпенсируют
разность фаз, и вектор \textbf{E} на выходе останется в первом
и третьем квадрантах. Если
же световой вектор перешёл в смежные квадранты, значит, эллипсы вращаются в одну сторону. 
\par После второго поляроида интенсивность света максимальна. Значит, две пластины усиливают друг друга, световой вектор перешёл в смежные квадранты, эллипсы вращаются в одну сторону.

\end{enumerate}

\subsection{Интерференция поляризованных лучей}

Расположим между скрещенными поляроидами мозаичную слюдяную пластинку. Она собрана из 4-х узких полосок слюды, лежащих по сторонам квадрата (две полоски «толщиной» $\lambda/4$ и по одной — $\lambda/2$ и $3\lambda/4$).
\par Вращаем пластинку: изменяется интенсивность света с периодичностью $\pi/4$
\par Вращаем второй поляроид: изменяется (инверсируется) цвет пластинок также с периодичностью  $\pi/4$. Расшифровка пластинки по длинам волн:

    \begin{table}[h]
    \centering
    \begin{center}
    \caption{Цвет ячеек и их толщина}
    \end{center}
    \vspace{0.1cm}
    \label{tab:my_label}
    \begin{tabular}{ |p{2.5cm}|p{2.5cm}|p{2.5cm}|}
 \hline
 $3\lambda/4$ зелёный & $\lambda/2$ пурпурный & $3\lambda/4$ зелёный\\
\hline
 $\lambda/4$ красный & - & $\lambda/4$ красный \\
\hline
 $\lambda$ жёлтый & $3\lambda/4$ синий & $\lambda$ жёлтый \\
\hline
 
\end{tabular}
\end{table}

\section{Вывод}
Поляризованный свет обладает большим числом свойств, которые можно
применять для исследования оптических характеристик различных приборов и
веществ.

\end{document}

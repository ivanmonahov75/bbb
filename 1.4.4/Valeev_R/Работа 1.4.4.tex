\documentclass[a4paper, 12pt]{article}%тип документа

%отступы
\usepackage[left=1cm,right=1cm,top=1cm,bottom=3cm,bindingoffset=0cm]{geometry}

%Русский язык
\usepackage[T2A]{fontenc} %кодировка
\usepackage[utf8]{inputenc} %кодировка исходного кода
\usepackage[english,russian]{babel} %локализация и переносы

%Вставка картинок
\usepackage{graphicx}
\graphicspath{{pictures/}}
\DeclareGraphicsExtensions{.pdf,.png,.jpg}

%Графики
\usepackage{pgfplots}
\pgfplotsset{compat=1.9}

%Математика
\usepackage{amsmath, amsfonts, amssymb, amsthm, mathtools}

%Заголовок
\author{Валеев Рауф Раушанович \\
группа 825}
\title{Работа 1.4.4 \\
Исследование свободных колебаний связанных маятников}
\begin{document}
\maketitle
\newpage
\textbf{Цель работы:} Изучение колебательной системы с двумя степенями свободы.

\textbf{В работе используются:} установка с двумя одинаковыми математическими маятниками, бифилярно подвешенными на натянутую горизонтально струну, секундомер, измерительная линейка. 

\textbf{Ход работы:}
\begin{enumerate}
\item Измеряем длину маятников $l = 43\pm 0,1$ см, расстояние между двумя неподвижными точками (на рис.1 $A$ и $B$) $b = 71\pm 0,1$ см и точками подвеса маятников $a = 23,7\pm 0,1$ см. Масса маятников $m_1 = 224,2$ г, $m_2 = 223,3$ г.
\begin{figure} [h]
\center{\includegraphics{144_1.jpg}}
\end{figure}
\item Измеряем периоды нормальных колебаний (мод). Для этого измеряем $T_1$ - переод колебаний в синофазе, то есть отклоняем маятники на угол $\approx 30^{ \circ}$ в одну сторону и считаем с помощью секундомера период, проводя измерения по 10 периодам. Проводим аналогичные измерения для противофазы, отклоняя маятники в разные стороны, измеряя период $T_2$. (табл. 1)
\item Измеряем периоды ($T_1^{'}$ и $T_2^{'}$) порциальных колебаний маятников отцепив 1 из них, и измеряя период другого. (табл. 1)
\item Проводим измерения раскачивания одного маятника другим, для этого в эксперименте пункта 2 отклоняем лишь 1 маятник и измеряем период биений $\tau$. (табл. 1)
\item Записываем все получившиеся данные и соответствующие ошибки в табл. 1.
\item Убеждаемся, что равенство (табл. 2)
\[\dfrac{1}{\tau} = \dfrac{1}{T_1} - \dfrac{1}{T_2}\]
\item Повторяем пункты 1 - 6 для других натяжений струны (табл. 1)
\item Строим график зависимости периода биения от натяжений.
\begin{figure} [h]
\center{\includegraphics{144_2.jpg}}
\end{figure}
\item Проводим сравнение полученных результатов с теоритическими расчетами по формуле 
\[ \tau \approx 6 \pi \dfrac{Ml}{ma} \sqrt{\dfrac{l}{g}} \] \\
\[ \sigma_{ \tau } = \tau \sqrt{\dfrac{3}{2} \left( \dfrac{\sigma_l}{l} \right) ^2 + \left( \dfrac{\sigma_a}{a} \right) ^2 } \]
\item Поскольку приведенная формула сходится в пределах 2 ошибок, то лабораторная работа выполнена весьма точно. 
\end{enumerate}
\begin{table}
\center{
\begin{tabular}{|c|c|c|}
\hline
& Значение, c/кг&$\sigma$, c/кг \\
\hline
$slope_{\tau_{theor}}$, c& 27,65 & 2,11 \\
\hline
$slope_{\tau_{pract}}$, c& 31,62& 0 \\
\hline
\end{tabular}
}
\end{table}
\newpage
\begin{center}
\begin{table}
\begin{center}
\begin{tabular}{|c|c|c|c|c|c|}
\hline
$M$, кг&\multicolumn{5}{|c|}{1,78} \\
\hline
&$T_1$, с&$T_2$, с&$T_1^{'}$, с&$T_2^{'}$, с& $ \tau $, с \\
\hline
1&1,396&1,356&1,37&1,37&61,5\\
\hline
2&1,386&1,365&1,37&1,37&63,4\\
\hline
3&1,41&1,34&1,36&1,36&58,71\\
\hline
среднее&1,397&1,354&1,37&1,37&61,2 \\
\hline
$\sigma = \sqrt{\sigma_{rnd}^2 + \sigma_{stat}^2}, \sigma_{stat} \approx 1 c$&0,1&0,1&0,1&0,1&1,49 \\
\hline
$M$, кг&\multicolumn{5}{|c|}{2,29} \\
\hline
&$T_1$, с&$T_2$, с&$T_1^{'}$, с&$T_2^{'}$, с& $ \tau $, с \\
\hline
1&1,37&1,35&1,37&1,37&74,8\\
\hline
2&1,373&1,352&1,37&1,37&76,53\\
\hline
3&1,365&1,346&1,36&1,36&73,2\\
\hline
среднее&1,369&1,349&1,37&1,37&74,84 \\
\hline
$\sigma = \sqrt{\sigma_{rnd}^2 + \sigma_{stat}^2}, \sigma_{stat} \approx 1 c$&0,1&0,1&0,1&0,1&1,27\\
\hline
$M$, кг&\multicolumn{5}{|c|}{2,79} \\
\hline
&$T_1$, с&$T_2$, с&$T_1^{'}$, с&$T_2^{'}$, с& $ \tau $, с \\
\hline
1&1,336&1,339&1,37&1,37&92,21 \\
\hline
2&1,34&1,363&1,37&1,37&92,42 \\
\hline
3&1,345&1,362&1,36&1,36&91,91 \\
\hline
среднее&1,34&1,355&1,37&1,37&92,18 \\
\hline
$\sigma = \sqrt{\sigma_{rnd}^2 + \sigma_{stat}^2}, \sigma_{stat} \approx 1 c$&0,1&0,1&0,1&0,1&1 \\
\hline
$M$, кг&\multicolumn{5}{|c|}{3,29} \\
\hline
&$T_1$, с&$T_2$, с&$T_1^{'}$, с&$T_2^{'}$, с& $ \tau $, с \\
\hline
1&1,378&1,351&1,37&1,37&104,4\\
\hline
2&1,381&1,361&1,37&1,37&102,3\\
\hline
3&1,38&1,345&1,36&1,36&98,24\\
\hline
среднее&1,380&1,352&1,37&1,37&101,65 \\
\hline
$\sigma = \sqrt{\sigma_{rnd}^2 + \sigma_{stat}^2}, \sigma_{stat} \approx 1 c$&0,1&0,1&0,1&0,1&1,78 \\
\hline
$M$, кг&\multicolumn{5}{|c|}{3,88} \\
\hline
&$T_1$, с&$T_2$, с&$T_1^{'}$, с&$T_2^{'}$, с& $ \tau $, с \\
\hline
1&1,366&1,338&1,37&1,37&120,7 \\
\hline
2&1,376&1,339&1,37&1,37&121,81 \\
\hline
3&1,368&1,346&1,36&1,36&116,12 \\
\hline
среднее&1,37&1,341&1,37&1,37&119,54 \\
\hline
$\sigma = \sqrt{\sigma_{rnd}^2 + \sigma_{stat}^2}, \sigma_{stat} \approx 1 c$&0,1&0,1&0,1&0,1&1,74 \\
\hline
\end{tabular}
\caption{Значения различных периодов при различной массе}
\end{center}
\center{
\begin{tabular}{|c|c|c|c|c|}
\hline
$M$, кг&$\dfrac{1}{T_{2}} - \dfrac{1}{T_{1}}$ 1/c&$\dfrac{1}{\tau}$ 1/c&$\Delta$ 1/c&$\varepsilon$ \\
\hline
1,79&0,023&0,017&0,006&35,2 \\
\hline
2,29&0,011&0,013&0,002&15,4 \\
\hline
2,79&0,008&0,011&0,003&27,3 \\
\hline
3,29&0,014&0,010&0,004&40 \\
\hline
3,88&0,015&0,009&0,006&66,7 \\
\hline
\end{tabular}
}
\caption{Проверка зависимости} 
\end{table}
\end{center}
\end{document}

\documentclass[12pt,a4paper]{article}
\usepackage{amsmath}
\usepackage{amsfonts}
\usepackage{amssymb}
\usepackage[utf8]{inputenc}
\usepackage[T1, T2A]{fontenc}
\usepackage[english, russian]{babel}
\usepackage{graphicx}
\usepackage[left=2cm,right=2cm,top=2cm,bottom=2cm]{geometry}
\usepackage{calc}
\usepackage{wrapfig}
\usepackage{setspace}
\usepackage{indentfirst}
\usepackage{subfigure}
\usepackage{booktabs}
\usepackage[table,xcdraw]{xcolor}


\title{
Отчет о выполнении лабораторной работы 2.3.1 \\
Получение и измерение вакуума
}

\author{Фокин Алексей, 922 группа}

\begin{document}
\maketitle

\paragraph{Цель работы:} 1) измерение объемов форвакуумной и высоковакуумной частей установки; 2) определение скорости откачки системы в стационарном режиме, а также по ухудшению и улучшению вакуума.
\paragraph{В работе используются:} вакуумная установка с манометрами: масляным, термопарным и ионизационным.

\section{Теоретическая справка}
\subsection*{Экспериментальная установка}
 В данной работе используются традиционные методы откачки механическим форвакуумным насосом до давления $10^{-2}$ торр и диффузионным масляным насосом до давления $10^{-4}$ торр. \\
 	Установка изготовлена из стекла,
 и состоит из форвакуумного баллона (ФБ), высоковакуумного диффузионного насоса (ВН), высоковакуумного баллона (ВБ), масляного (М) и ионизационного (И) манометров, термопарных манометров ($\text{М}_1$ и $\text{М}_2$), форвакуумного насоса (ФН) и соединительных кранов ($K_1, K_2,..., K_6$) (рис. 1). Кроме того, в состав установки входят: вариатор
 (автотрансформатор с регулируемым выходным напряжением), или
 реостат и амперметр для регулирования тока нагревателя диффузионного насоса. \\
  \begin{figure}[!h]
 	\centering
 	\includegraphics[width=0.5\linewidth]{"../../../../../../Users/ПК/Desktop/Учеба/Лабы/Терма/2.3.1/Схема установки"}
 	\caption[]{Схема установки}
 	\label{fig:Схема установки}
 \end{figure}
 Все краны вакуумной установки стеклянные. Стенки кранов тонкие, пробки кранов полые и составляют одно целое с рукоятками. Пробки кранов притерты к корпусам. Для герметизации используется вакуумная смазка. \\
 Устройство и принцип действия \textit{форвакуумного насоса} схематически, но довольно ясно изображены на рис 2. В положениях <<а>> и <<б>> пластина <<А>> засасывает разреженный воздух из откачиваемого объёма, а пластина <<Б>> вытесняет ранее захваченный воздух в атмосферу. В положениях <<в>> и <<г>> пластины поменялись ролями.
\begin{figure}[!h]
	\centering
	\includegraphics[width=0.9\linewidth]{"../../../../../../Users/ПК/Desktop/Учеба/Лабы/Терма/2.3.1/устройство фв насоса"}
	\caption[]{Схема действия ротационного двухпластинчатого форвакуумного насоса}
	\label{fig:Схема ФВ насоса}
\end{figure}
Устройство и принцип действия \textit{диффузионного насоса} схематически изображены на рис 2. Такой насос работает в тысячи раз быстрее форвакуумного. Его действие основано на диффузии. Масло, налитое в сосуд А, подогревается электрической печкой. Пары масла поднимаются по трубке Б и вырываются из сопла В. Струя паров увлекает молекулы газа, которые поступают из откачиваемого сосуда через трубку ВВ. В трубке Г мало осаждается и стекает вниз. Оставшийся газ, выходя в трубку ФВ, откачивается форвакуумным насосом. \\
Диффузионный насос работает наиболее эффективно, когда длина свободного пробега молекул примерно равна ширине кольцевого зазора между соплом В и стенками трубки ВВ. Давление насыщенных паров масла при рабочей температуре, создаваемой обогревателем сосуда А, много больше $5\cdot 10^{-2}$ торр, поэтому пары масла создают плотную струю, увлекающую с собой молекулы газа.
\begin{figure}[!h]
	\centering
	\includegraphics[width=0.4\linewidth]{"../../../../../../Users/ПК/Desktop/Учеба/Лабы/Терма/2.3.1/устройство вв насоса"}
	\caption[]{Схема работы диффузионного насоса}
	\label{fig:Схема ВВ насоса}
\end{figure}
 Диффузионный насос, используемый в нашей установке (см. рис 1) имеет две ступени и соответственно два сопла. Одно сопло вертикальное (первая ступень), второе горизонтальное (вторая ступень). За второй ступенью имеется ещё одна печь, но пар из этой печи поступает не в сопло, а по тонкой трубке подводится ближе к печке первой ступени. Эта печь осуществляет фракционирование масла. Легколетучие фракции масла, испаряясь, поступают в первую ступень, обогащая её. По этой причине плотность струи первой ступени выше, и эта ступень начинает откачивать при более высоком давлении в форвакуумной части. Вторая ступень обогащается малолетучими фракциями масла. Плотность струи второй ступени меньше, но меньше и давление насыщенных паров. Соответственно, в откачиваемый объем поступает меньше паров масла, и его удаётся откачать до более высокого вакуума.  \\
 \begin{wrapfigure}{r}{50mm}
 	\begin{center}
 		\includegraphics[width=0.9\linewidth]{"../../../../../../Users/ПК/Desktop/Учеба/Лабы/Терма/2.3.1/термопара"}
 		\caption{Схема термопарного манометра с лампой ЛТ-2}
 		\label{fig:Схема термопары}
 	\end{center}
 \end{wrapfigure}
 \textit{Термопарный манометр.} Чувствительным элементом манометра является платиново-родиевая термопара, спаянная с никелевой нитью накала и заключённая в стеклянный баллон. Устройство термопары пояснено на рис. 4. По нити накала НН пропускается ток постоянной величины. Для установки тока служит потенциометр R, расположенный на передней панели вакуумметра. Термопара ТТ присоединяется к милливольтметру, показания которого определяются температурой нити накала и зависят от отдачи тепла в окружающее пространство. \\
 Потери тепла определяются теплопроводностью нити и термопары, теплопроводностью газа, переносом тепла конвективными потоками газа внутри лампы, и теплоизлучением нити (инфракрасное тепловое излучение). В обычном режиме лампы основную роль играет теплопроводность газа. При давлениях, не меньших 1 торр, теплопроводность газа, а вместе с ней и ЭДС термопары практически не зависят от давления газа, и прибор не работает. \\
 При улучшении вакуума средний свободный пробег молекул становится сравнимым с диаметром нити, теплоотвод падает, и температура спая возрастает. При вакууме порядка $10^{-3}$ торр теплоотвод, осуществляемый газом, становится сравнимым с другими потерями тепла, и температура становится практически постоянной. Градуировочная кривая термопары приведена на рис. 5. \\
 \begin{figure}[!h]
 	\centering
 	\includegraphics[width=0.4\linewidth]{"../../../../../../Users/ПК/Desktop/Учеба/Лабы/Терма/2.3.1/градуировочная кривая"}
 	\caption[]{Градуировочная кривая термопары ЛТ-2}
 	\label{fig:Градуировочная кривая}
 \end{figure}
\newpage
 \begin{wrapfigure}{r}{60mm} %если не лезет, но с новой стр
	\begin{center}
		\includegraphics[width=60mm]{"../../../../../../Users/ПК/Desktop/Учеба/Лабы/Терма/2.3.1/лампа"}
		\caption{Схема ионизационной лампы ЛТ-2}
		\label{fig:лампа}
	\end{center}
\end{wrapfigure}
\textit{Ионизационный манометр.} Схема ионизационного манометра изображения на рисунке 6. Он представляет собой трехэлектродную лампу. Электроны испускаются раскалённым катодом и увлекаются электрическим полем к аноду, имеющему вид редкой спирали. Проскакивая за её витки, электроны замедляются полем коллектора и возвращаются к аноду. Прежде чем осесть на аноде, они успевают много раз пересечь пространство между катодом и коллектором. На своём пути электроны ионизуют молекулы газа. Ионы, образовавшиеся между анодом и коллектором, притягиваются полем коллектора и определяют его ток. \\
  Накалённый катод ионизационного манометра перегорает, если давление в системе превышает $10^{-3}$ торр, поэтому перед его включением необходимо проверить давление термопарным манометром. \\
 \subsection*{Процесс откачки}
Опишем процесс откачки математически: 
Пусть W --- объем газа, удаляемого из сосуда при данном давлении за единицу времени, $Q_i$ для различных значений i обозначим различные притоки газа в сосуд (в единицах PV), такие как течи извне $Q_\text{и}$, десорбция с поверхностей внутри сосуда $Q_\text{д}$, обратный ток через насос $Q_\text{н}$. Тогда, приравнивая убыль газа из сосуда (с точностью до $RT/\mu$) в единицу времени $-VdP$ и сумму перечисленных токов? имеем:
 \begin{equation}
 	-VdP = (PW - \sum_i Q_i)dt
 \end{equation}
 При достижении предельного вакуума устанавливается давление $P_{\text{пр}}$, и $dP = 0$. Тогда
 \begin{equation}
 	 W = ( \sum_i Q_i )/P_{\text{пр}}
 \end{equation}
 Поскольку обычно $Q_\text{и}$ постоянно, а $Q_\text{н}$ и $Q_\text{д}$ слабо зависят от времени, также считая постоянной W, можем проинтегрировать (1) и получить:
 \begin{equation}
 	P - P_{\text{пр}} = (P_0 - P_{\text{пр}})\exp(-\frac{W}{V}t)
 \end{equation}
Полная скорость откачки $W$, собственная скорость откачки насоса $W_{\text{н}}$ и проводимости элементов системы $C_1, C_2,...$ соотносятся согласно формуле (4), и это учтено в конструкции установки.
 \begin{equation}
 \frac{1}{W} = \frac{1}{W} + \frac{1}{C_1} + \frac{1}{C_2} + ...
\end{equation}

\subsection*{Течение газа через трубу}
Характер течения газа существенно зависит от соотношения между размерами системы и длиной свободного пробега молекул. При атмосферном и форвакуумном давлениях  длина свободного пробега меньше диаметра трубок, и течение газа определяется его вязкостью, т.е. взаимодействием молекул. При высоком вакууме течение существеннее определяется взаимодействием со стенками \\
Для количества газа, протекающего через трубу длины $l$ и радиуса $r$ в условиях высокого вакуума, справедлива формула:
\begin{equation}
	\frac{d(PV)}{dt} = \frac{4}{3}r^3\sqrt{\frac{2\pi RT}{\mu}}\frac{P_2 - P_1}{l}
\end{equation}
Если труба соединяет насос установку, то давлением $P_1$ у насоса можно пренебречь. Давление в сосуде $P = P_2$. Тогда имеем:
\begin{equation}
C_\text{тр} = \left(\frac{dV}{dt}\right)_\text{тр} = \frac{4r^3}{3l}\sqrt{\frac{2\pi RT}{\mu}}
\end{equation}
Для пропускной способности отверстий имеется формула
\begin{equation}
C_\text{отв} = \left(\frac{dV}{dt}\right)_\text{отв} = S\frac{\bar{\upsilon}}{4}
\end{equation}
Для воздуха при комнатной температуре $\bar{\upsilon}/4 = 110~\text{м/с} = 11~\text{л/c}\cdot\text{см}^2$.
\section{Ход работы}

\subsection*{Измерение объёмов форвакуумной и высоковакуумной частей установки}
	\begin{enumerate}
		\item Проверяем, что $K_4$ открыт, впускаем в установку атмосферный воздух через краны $K_1$ и $K_2$. <<Запираем>> в капилляре атмосферный воздух кранами $K_5$ и $K_6$. Объем капилляра в используемой установке: $$V_\text{к} = 50~ \text{см}^3.$$
		\item Закрываем $K_1$ и $K_2$, включаем форвакуумный насос и даём ему откачать себя. Подключаем установку к насосу краном $K_2$. Откачиваем установку до $10^{-2}$ торр. Отсоединяем установку краном $K_2$, и оставляем насос работать <<на себя>>. Перекрываем  $K_3$, отделяя высоковакуумною часть установки. Закрываем  $K_4$, чтобы привести в готовность масляный манометр.
		\item Открываем  $K_5$, чтобы <<запертый>> ранее воздух заполнил форвакуумную часть установки, снимаем давление с помощью вакуумного манометра, измерив разность высот столбиков масла (приводим результаты и повторного измерения):
		$$
		\Delta h_1 = (15.1\pm0.1) ~\text{мм};\quad \Delta h_2 (15.2\pm0.1) ~\text{мм}
		$$
		Погрешность измерения величин определяется ценой деления шкалы манометра и способностью разглядеть показания.
		\item Имея в виду, что плотность масла в манометре равна 885 г/л, и считая, что установившееся давление много больше форвакуумного, получаем:
		$$
		P_1 = (1.31\pm0.01)~\text{Па};\quad P_2 = (1.32\pm0.01)~\text{Па}
		$$
		Пользуясь законом Бойля-Мариотта (т.к. расширение газа изотермическое), используя среднее значение измеренного давления, получаем
		$$
		V_\text{ФВ} = (3750\pm20)~\text{см}^3
		$$
		\item Аналогично, открыв кран $K_3$, получив значения разности высот на манометре
		$$
		\Delta h_1 = (10.3\pm0.1) \text{мм}; \Delta h_2 (10.3\pm0.1) \text{мм},
		$$
		Получаем объем высоковакуумной части установки, вычтя из полученного законом Бойля-Мариотта объёма двух частей установки объем измеренной ранее части (погрешности складываются):
		$$
		V_\text{ВВ} = (1760\pm60)~\text{см}^3
		$$
		\item Открываем кран $K_4$.
	\end{enumerate}

\subsection*{Получение высокого вакуума}
	\begin{enumerate}
	\item Откачиваем установку ФВ насосом.
	\item Включаем термопарные манометры, устанавливаем их токи согласно паспортам. Переключаем прибор в режим измерения ЭДС и определяем давление в установке по градуировочной кривой (рис. 5)
	\item По достижении форвакуума закрываем $K_5$ и начинаем откачку высоковакуумного баллона с помощью диффузионного насоса, для этого:\\
	3.1. На передней панели источника питания, с помощью которого подогревается масло в насосе, все четыре ручки переводим на ноль.\\
	3.2. Включаем источник, ручками 2 и 4 устанавливаем ток $I = 0,6A$. Ждём 5 минут, чтобы масло прогрелось, после чего устанавливаем ток $1,15A$. \\
	3.3. По термопаре $M_2$ контролируем откачку. По достижении ЭДС в 10mV смотрим на кипение масла и считаем капли, стекающие из сопла второй ступени. Убеждаемся в готовности, т.к. насчитали 11 капель в минуту.
	\item При выключенной ионизационной лампе, вставив предохранитель, ставим переключатель <<Род работы>> в положение <<Обезгаживание>> на 10 минут.
	\item Переключатель <<Множитель шкалы>> ставим в положение <<Установка нуля>>, <<Род работы>> в положение <<Установка эмиссии>>, и ручку <<Установка эмиссии>> ставим в крайнее левое положение.
	\item Приступаем к включению ионизационной лампы: \\
	6.1. Тумблер <<накал>> в положение вкл. \\
	6.2. <<Род работы>> в положение <<Измерение>>, ручкой <<Установка нуля>> устанавливаем микроамперметр на нуль.\\
	6.3. <<Род работы>> в положение <<Прогрев>> на 10 минут. \\
	6.4. <<Род работы>> в положение <<Установка эмиссии>>, и ручкой <<Установка эмиссии>> устанавливаем ток эмиссии в 50 мкА. \\ 
	6.5. <<Род работы>> в положение <<Обезгаживание>> на 10 минут.
	6.6. <<Род работы>> в положение <<Измерение>> \\
	6.7. Переключатель <<Множитель шкалы>> ставим в положение $10^{-1}$. 
	\item Измеряем давление с помощью микроамперметра. Так как переключатель <<Множитель шкалы>> в положении $10^{-1}$, а постоянная ионизационного манометра $С = 100$ мм.рт.ст./А, давление будет определяться как $P = 10^{-5} I$, где I --- показания микроамперметра в делениях. \\
	$$P_{\text{пр}} = 1,1\cdot10^{-4} ~\text{торр}$$	
\end{enumerate}

\subsection*{Измерение скорости по ухудшению и улучшению вакуума}
\begin{enumerate}
	\item Закрываем кран $K_3$, отключая тем самым откачку вакуума и записываем на видео изменения показаний микроамперметра, пока вакуум не ухудшится до $6\cdot10^{-4}$ торр. Затем открываем $K_3$ и так же записываем улучшение вакуума. Приводим результаты повторных измерений в таблице 1 и на графиках (рис. 7 и 8).
	\begin{table}[!h]
		\centering
		\resizebox{0.9\textwidth}{!}{%
			\begin{tabular}{|c|l|cl|c|c|c|c|cc}
				\hline
				\multicolumn{2}{|c|}{{\color[HTML]{000000} \textbf{Улучшение,   1}}} & \multicolumn{2}{c|}{{\color[HTML]{000000} \textbf{Улучшение, 2}}} & \multicolumn{2}{c|}{{\color[HTML]{000000} \textbf{Ухудшение, 1}}} & \multicolumn{2}{c|}{{\color[HTML]{000000} \textbf{Ухудшение, 2}}} & \multicolumn{2}{c|}{{\color[HTML]{000000} \textbf{Ухудшение, 3}}} \\ \hline
				{\color[HTML]{000000} \textbf{t, с}} & \multicolumn{1}{c|}{{\color[HTML]{000000} \textbf{p, торр}}} & \multicolumn{1}{l|}{{\color[HTML]{000000} \textbf{t, с}}} & {\color[HTML]{000000} \textbf{p, торр}} & \multicolumn{1}{l|}{{\color[HTML]{000000} \textbf{t, с}}} & {\color[HTML]{000000} \textbf{p, торр}} & \multicolumn{1}{l|}{{\color[HTML]{000000} \textbf{t, с}}} & \multicolumn{1}{l|}{{\color[HTML]{000000} \textbf{p, торр}}} & \multicolumn{1}{l|}{{\color[HTML]{000000} \textbf{t, с}}} & \multicolumn{1}{l|}{{\color[HTML]{000000} \textbf{p, торр}}} \\ \hline
				{\color[HTML]{000000} 0} & {\color[HTML]{000000} 0,0006} & \multicolumn{1}{c|}{{\color[HTML]{000000} 0}} & {\color[HTML]{000000} 0,000615} & {\color[HTML]{000000} 0} & {\color[HTML]{000000} 0,00009} & {\color[HTML]{000000} 0} & {\color[HTML]{000000} 0,00008} & \multicolumn{1}{c|}{{\color[HTML]{000000} 0}} & \multicolumn{1}{c|}{{\color[HTML]{000000} 0,0001}} \\ \hline
				{\color[HTML]{000000} 0,167} & {\color[HTML]{000000} 0,00058} & \multicolumn{1}{c|}{{\color[HTML]{000000} 0,33}} & {\color[HTML]{000000} 0,0006} & {\color[HTML]{000000} 5,5} & {\color[HTML]{000000} 0,00012} & {\color[HTML]{000000} 4} & {\color[HTML]{000000} 0,00012} & \multicolumn{1}{c|}{{\color[HTML]{000000} 5}} & \multicolumn{1}{c|}{{\color[HTML]{000000} 0,00015}} \\ \hline
				{\color[HTML]{000000} 0,33} & {\color[HTML]{000000} 0,000555} & \multicolumn{1}{c|}{{\color[HTML]{000000} 0,9}} & {\color[HTML]{000000} 0,00056} & {\color[HTML]{000000} 10} & {\color[HTML]{000000} 0,00016} & {\color[HTML]{000000} 12} & {\color[HTML]{000000} 0,00018} & \multicolumn{1}{c|}{{\color[HTML]{000000} 12}} & \multicolumn{1}{c|}{{\color[HTML]{000000} 0,0002}} \\ \hline
				{\color[HTML]{000000} 0,5} & {\color[HTML]{000000} 0,00054} & \multicolumn{1}{c|}{{\color[HTML]{000000} 1,5}} & {\color[HTML]{000000} 0,00052} & {\color[HTML]{000000} 16} & {\color[HTML]{000000} 0,0002} & {\color[HTML]{000000} 20} & {\color[HTML]{000000} 0,00024} & \multicolumn{1}{c|}{{\color[HTML]{000000} 18}} & \multicolumn{1}{c|}{{\color[HTML]{000000} 0,00025}} \\ \hline
				{\color[HTML]{000000} 0,75} & {\color[HTML]{000000} 0,00052} & \multicolumn{1}{c|}{{\color[HTML]{000000} 2}} & {\color[HTML]{000000} 0,00046} & {\color[HTML]{000000} 21} & {\color[HTML]{000000} 0,00024} & {\color[HTML]{000000} 26} & {\color[HTML]{000000} 0,00028} & \multicolumn{1}{c|}{{\color[HTML]{000000} 25}} & \multicolumn{1}{c|}{{\color[HTML]{000000} 0,0003}} \\ \hline
				{\color[HTML]{000000} 1} & {\color[HTML]{000000} 0,0005} & \multicolumn{1}{c|}{{\color[HTML]{000000} 2,4}} & {\color[HTML]{000000} 0,00042} & {\color[HTML]{000000} 26} & {\color[HTML]{000000} 0,00028} & {\color[HTML]{000000} 31} & {\color[HTML]{000000} 0,00032} & \multicolumn{1}{c|}{{\color[HTML]{000000} 32}} & \multicolumn{1}{c|}{{\color[HTML]{000000} 0,00035}} \\ \hline
				{\color[HTML]{000000} 1,25} & {\color[HTML]{000000} 0,00047} & \multicolumn{1}{c|}{{\color[HTML]{000000} 3}} & {\color[HTML]{000000} 0,00038} & {\color[HTML]{000000} 32} & {\color[HTML]{000000} 0,00032} & {\color[HTML]{000000} 37} & {\color[HTML]{000000} 0,00036} & \multicolumn{1}{c|}{{\color[HTML]{000000} 39}} & \multicolumn{1}{c|}{{\color[HTML]{000000} 0,0004}} \\ \hline
				{\color[HTML]{000000} 1,5} & {\color[HTML]{000000} 0,00043} & \multicolumn{1}{c|}{{\color[HTML]{000000} 4}} & {\color[HTML]{000000} 0,00032} & {\color[HTML]{000000} 40} & {\color[HTML]{000000} 0,00038} & {\color[HTML]{000000} 43} & {\color[HTML]{000000} 0,0004} & \multicolumn{1}{c|}{{\color[HTML]{000000} 46}} & \multicolumn{1}{c|}{{\color[HTML]{000000} 0,00045}} \\ \hline
				{\color[HTML]{000000} 2} & {\color[HTML]{000000} 0,0004} & \multicolumn{1}{c|}{{\color[HTML]{000000} 5}} & {\color[HTML]{000000} 0,00028} & {\color[HTML]{000000} 47} & {\color[HTML]{000000} 0,00042} & {\color[HTML]{000000} 49} & {\color[HTML]{000000} 0,00044} & \multicolumn{1}{c|}{{\color[HTML]{000000} 52}} & \multicolumn{1}{c|}{{\color[HTML]{000000} 0,0005}} \\ \hline
				{\color[HTML]{000000} 3} & {\color[HTML]{000000} 0,00032} & \multicolumn{1}{c|}{{\color[HTML]{000000} 6}} & {\color[HTML]{000000} 0,00024} & {\color[HTML]{000000} 52} & {\color[HTML]{000000} 0,00046} & {\color[HTML]{000000} 54} & {\color[HTML]{000000} 0,00048} & \multicolumn{1}{c|}{{\color[HTML]{000000} 60}} & \multicolumn{1}{c|}{{\color[HTML]{000000} 0,00055}} \\ \hline
				{\color[HTML]{000000} 4} & {\color[HTML]{000000} 0,00027} & \multicolumn{1}{c|}{{\color[HTML]{000000} 8}} & {\color[HTML]{000000} 0,00019} & {\color[HTML]{000000} 57} & {\color[HTML]{000000} 0,0005} & {\color[HTML]{000000} 60} & {\color[HTML]{000000} 0,00052} & \multicolumn{1}{c|}{{\color[HTML]{000000} 68}} & \multicolumn{1}{c|}{{\color[HTML]{000000} 0,0006}} \\ \hline
				{\color[HTML]{000000} 6} & {\color[HTML]{000000} 0,0002} & \multicolumn{1}{c|}{{\color[HTML]{000000} 12}} & {\color[HTML]{000000} 0,00014} & {\color[HTML]{000000} 66} & {\color[HTML]{000000} 0,00055} & {\color[HTML]{000000} 68} & {\color[HTML]{000000} 0,00056} & {\color[HTML]{000000} } & {\color[HTML]{000000} } \\ \cline{1-8}
				{\color[HTML]{000000} 9} & {\color[HTML]{000000} 0,00015} & {\color[HTML]{000000} } & {\color[HTML]{000000} } & {\color[HTML]{000000} 74} & {\color[HTML]{000000} 0,0006} & {\color[HTML]{000000} 74} & {\color[HTML]{000000} 0,0006} & {\color[HTML]{000000} } & {\color[HTML]{000000} } \\ \cline{1-2} \cline{5-8}
			\end{tabular}%
		}
	\caption{Зависимости давления от времени}	
	\end{table}

	\begin{figure}[!h]
		\begin{center}
			\subfigure[Измерение 1]{\includegraphics[width=0.45\textwidth]{"C:/Users/ПК/Desktop/Учеба/Лабы/Терма/2.3.1/експ1"}}
			\subfigure[Измерение 2]{\includegraphics[width=0.45\textwidth]{"C:/Users/ПК/Desktop/Учеба/Лабы/Терма/2.3.1/експ2"}}
		\end{center}
		\caption{Зависимость давления от времени по улучшении вакуума}
	\end{figure}

	\begin{figure}[!h]
		\begin{center}
			\subfigure[Измерение 1]{\includegraphics[width=0.3\textwidth]{"C:/Users/ПК/Desktop/Учеба/Лабы/Терма/2.3.1/линия 1"}}
			\subfigure[Измерение 2]{\includegraphics[width=0.3\textwidth]{"C:/Users/ПК/Desktop/Учеба/Лабы/Терма/2.3.1/линия 2"}}
			\subfigure[Измерение 3]{\includegraphics[width=0.3\textwidth]{"C:/Users/ПК/Desktop/Учеба/Лабы/Терма/2.3.1/линия 3"}}
		\end{center}
		\caption{Зависимость давления от времени по улучшении вакуума}
	\end{figure}
	\item
	Рассчитав коэффициенты наклона графиков 7(а) и 7(б) и зная объем высоковакуумной части установки, получим скорость откачки W диффузионного насоса, сравнив графики с зависимостью (4). Считаем $$W = -\bar{a}\cdot V, \quad\varepsilon_W^2 = \varepsilon_{\bar{a}}^2 + \varepsilon_V^2$$,  где $\bar{a}$ --- среднее коэффициентов наклона из зависимостей 7(а) и 7(б). Имеем:
	$$W = (0,461\pm0,016) ~\text{л/с}$$
	\item
	Имея в виду соотношения (1) для случая ухудшения вакуума (без откачки), оценим $Q_\text{н}$ c помощью полученных зависимостей 8(а, б, в). Считаем $$\frac{dP}{dt} = \bar{a}$$ где $\bar{a}$ --- среднее коэффициентов наклона из зависимостей 8(а), 8(б), 8(в). Имеем:
	$$Q_\text{н} + Q_\text{д} = (1,26\pm0,04)\cdot 10^{-5} ~\text{торр}\cdot\text{л/c}$$
	$Q_\text{д}$ обычно порядка $10^{-8}$, поэтому можно считать $Q_\text{н} + Q_\text{д} \approx Q_\text{н}$. Таким образом,
	$$Q_\text{н} + Q_\text{д} \approx 1,26\cdot 10^{-5} ~\text{торр}\cdot\text{л/c}$$
	\item Оценим пропускную способность трубы от вакуумного баллона, имея в виду порядки её диаметра и длины и размерного множителя $$d \sim 10^{-2}~\text{м},\quad L \sim 1 ~\text{м},\quad \sqrt{\frac{RT}{\mu}} \sim 500 ~\text{м/с},$$ используя формулу (6) имеем:
	$$C_\text{тр} \sim 1 ~\text{л/с},$$
	что отлично согласуется с полученным ранее значением W.
	\item
	Рассчитаем производительность насоса ещё одним способом: создав искусственную течь. Открываем кран $K_6$ при включённом  насосе и измеряем давление, установившееся при течи. Оно равно $$P_\text{уст} = 1,9 \cdot 10^{-4} ~\text{торр}.$$
	Запишем (2) для данного случая:
		$$P_\text{пр}W = Q_1, \quad P_\text{уст}W = Q_1 + \frac{(PV)_\text{капилляр}}{dt}$$
	С учётом (6) получаем
	$$(P_\text{уст} - P_\text{пр})W = \frac{4}{3}(d/2)^3\sqrt{\frac{2\pi RT}{\mu}}\frac{P_\text{фв}}{L},$$
	где d и L --- диаметр и длина капилляра, равные
	$$d = 9 ~\text{мм},\quad L = 63 ~\text{мм}$$
	Получаем:
	$$W = 0,23 ~\text{л/c}$$
	Отличается почти ровно в два раза от полученной ранее. Вероятно, потому что теперь течение газа определяется пропускной способностью двух труб, соединённых последовательно (формула (5)). Судя по всему, проводимости трубки от ВВ баллона и капилляра сравнимы.
\end{enumerate}

\subsection*{Выключение установки}
\begin{enumerate}
	\item Выключаем накал ионизационного манометра, даем ему остыть, отключаем предохранитель.
	\item Выключаем подогрев диффузионного насоса, даем маслу остыть.
	\item Отключаем установку от ФВ насоса.
	\item Выключаем ФВ насос, ждем, соединяем его с атмосферой.
	\item Выключаем вакуумметры.
\end{enumerate}


\section{Выводы}
\begin{enumerate}
	\item В ходе данной работы было проверено несколько методик по измерению производительности высоковакуумного насоса.
	\item Проверены теоретические зависимости, связанные с течением газа (рис. 7 и 8)
	\item Измерено значение производительности насоса с точностью $\varepsilon = 0,03$
\end{enumerate}
\end{document}

\documentclass[a4paper, 12pt]{article}%тип документа

%отступы
\usepackage[left=2cm,right=2cm,top=2cm,bottom=3cm,bindingoffset=0cm]{geometry}

%Русский язык
\usepackage[T2A]{fontenc} %кодировка
\usepackage[utf8]{inputenc} %кодировка исходного кода
\usepackage[english,russian]{babel} %локализация и переносы

%Вставка картинок
\usepackage{wrapfig}
\usepackage{graphicx}
\graphicspath{{pictures/}}
\DeclareGraphicsExtensions{.pdf,.png,.jpg}

%оглавление
\usepackage{titlesec}
\titlespacing{\chapter}{0pt}{-30pt}{12pt}
\titlespacing{\section}{\parindent}{5mm}{5mm}
\titlespacing{\subsection}{\parindent}{5mm}{5mm}
\usepackage{setspace}

%Графики
\usepackage{multirow}
\usepackage{pgfplots}
\pgfplotsset{compat=1.9}

%Математика
\usepackage{amsmath, amsfonts, amssymb, amsthm, mathtools}

%Заголовок
\author{Борисов Владимир \\
группа 825}
\title{\textbf{Работа 2.2\\Изучение спектров атомов водорода и молекулы йода}}
\newtheorem{task}{Задача}
\begin{document}
\maketitle
\section*{Теория}
Атом водорода является простейшей атомной системой; для него уравнение Шредингера можно решить точно. 

С одной стороны задача об относительном движении электрона и ядра может быть легко сведена к задаче о движении частицы с эффективной массой $\mu = m_e M/(m_e + M)$ в кулоновском поле $-Ze^2/r$. Однако это решение не является простым, так как длины волн спектральных линий описываются формулой 
\begin{equation}
\dfrac{1}{\lambda_{mn}} = RZ^2\left(\frac{1}{n^2} - \frac{1}{m^2}\right),
\end{equation}
где $R$ --- константа Ридберга, а $m$ и $n$ --- целые числа.

Физический смысл этой формулы объясняется тремя постулатами Бора:
\textit{
\begin{enumerate}
\item из всех возможных с точки зрения классической физики орбит в атоме осуществляются только некоторые стационарные орбиты,при движении по которым, вопреки представлениям классической электродинамики, электрон не излучает энергии;
\item из всех возможных орбит в атоме осуществляются только те, для которых момент количества движения равен целому кратному величины постоянной Планка $\hbar = h/(2\pi)$ т.е.
\begin{equation}
L = n \hbar
\end{equation}
\item излучение или поглощение энергии происходит при переходе атома из одного стационарного состояния в другое, а частота излучаемого света связана с разностью энергий атома в стационарных состояниях соотношением 
\begin{equation}
h \nu = E_2 - E_1,
\end{equation}
где $\nu$ --- частота излучаемой линии.
\end{enumerate}
}

Из этих постулатов и кулоновского взаимодействия легко понять, какие это энергетические уровни, а именно
\begin{equation}
E_n = - \frac{2\pi^2m_ee^4Z^2}{h^2}\frac{1}{n^2}
\end{equation}

А из формулы $(4)$ мы легко можем определить частоты излучения. 

Из рис.1 видно, что линии в спектре водорода можно расположить по сериям; для всех линий $n$ постоянно, а $m$ меняется от $n+1$ до $\infty$. 

В данной работе мы изучаем серию Бальмера, линии которой лежат в видимой области. 

Для серии Бальмера $n=2$, а $m = 3, 4, 5, 6$. Эти линии обозначаются $H_{\alpha}, H_{\beta}, H_{\gamma}, H_{\delta}$.

Проводя нехитрые преобразования можно вычислить, что энергия основного состояния водородоподобного атома равна
\begin{equation}
E = -RZ^2
\end{equation}

И подобными же преобразованиями можно получить, что энергия возбужденного атома равна 
\begin{equation}
E_n = -R\frac{Z^2}{n^2}
\end{equation}

Поскольку по факту, у нас движение зависит от массы ядра, то имеет место так называемый \textit{изотопический сдвиг}, то есть различие в спектральных линиях у различных ядер, легко показать, что она выражается как
\begin{equation}
\frac{\Delta \lambda}{\lambda} = \frac{m_e}{m_p}\frac{A_D - A_H}{A_DA_H} \approx \frac{m_e}{2M_H}
\end{equation}
\newpage

Для измерения длин волн в работе используется стеклянно-призменный монохроматор-спектрометр. 
Основные его элементы:

1) Входная щель с микрометрическим винтом, 

2) Коллиматорный объектив с микрометрическим винтом,
 
3) Спектральная призма, Поворотный столик, вращающийся при помощи вертикального микрометрического винта, 

4) Зрительная труба, 

5) Оптическая скамья.

Молекулярный спектр йода можно наблюдать при помощи 

1) источника сплошного спектра --- лампу накаливания;

2) поглощающую среду --- кювету с йодом;

3) спектральный прибор --- монохроматор.

Кювета подогревается нихромовой спиралью, подключенной вместе с лампой накаливания к блоку питания, линза используется как конденсатор.

В результате подогревания кристаллы йода частично возгоняются, образуя пары с легкой фиолетовой окраской.
\newpage
\section*{Ход работы}
\subsection*{Изучение спектра атомов водорода}
Для начала проведем градуировку барабана монохроматора по спектрам ртутной и неоновой ламп. 
\begin{table}[h]
\begin{center}
\begin{tabular}{|c|c|c|c|}
\hline
Барабан, $^{0}$ & $\Delta$ Барабана, $^{0}$ & $\lambda,$ \AA & $\Delta \lambda,$ \AA \\ \hline
636                 & 1                             & 4047           & 5                     \\ \hline
1192                & 1                             & 4358           & 5                     \\ \hline
1857                & 1                             & 4916           & 5                     \\ \hline
2282                & 1                             & 5461           & 5                     \\ \hline
2461                & 1                             & 5770           & 5                     \\ \hline
2472                & 1                             & 5791           & 5                     \\ \hline
2506                & 1                             & 5945           & 5                     \\ \hline
2524                & 1                             & 5976           & 5                     \\ \hline
2552                & 1                             & 6030           & 5                     \\ \hline
2566                & 1                             & 6074           & 5                     \\ \hline
2592                & 1                             & 6096           & 5                     \\ \hline
2610                & 1                             & 6143           & 5                     \\ \hline
2622                & 1                             & 6164           & 5                     \\ \hline
2640                & 1                             & 6217           & 5                     \\ \hline
2674                & 1                             & 6234           & 5                     \\ \hline
2915                & 1                             & 6907           & 5                     \\ \hline
\end{tabular}

\caption{Данные для градуировки}

\end{center}
\end{table}

Погрешность барабана берется из половины цены деления, а погрешность длины волны мы берем из того факта, что при градуировке неоновой лампой спектр был частый, из-за чего были не очень хорошо различимы различные оттенки одного цвета, из-за чего берем за погрешность длины волны максимальное расстояние между соседними спектрами. Так же на график не имеет смысла наносить кресты ошибок, поскольку относительная погрешность меньше сотой процента.


Градуировка получается нелинейная, в итоге мы ее аппроксимируем полиномом 
\[y = A + Bx + Cx^2 + Dx^3,\]
где $A = 3600\pm160$, $B = 0,9\pm0,3$, $C = (-4\pm2) \cdot 10^{-4}$, $D = (1,8 \pm 0,4) \cdot 10^{-7}$.
\newpage
Далее померяем спектры водорода и, зная соответствующие длины волн получим их из графика градуировки, занесем данные в таблицу. Так же для каждого $H$ посчитаем $R_H$, посчитаем среднее и занесем все в таблицу. 

$R_H$ получаем из формулы $(1)$, замечая, что для всех $H$ $n = 2$, а $m = 3, 4, 5$ и $Z=1$ для водорода.
\begin{table}[h]
\begin{center}
\begin{tabular}{|c|c|c|c|}
\hline
                            & $H_{\gamma}$ & $H_{\beta}$ & $H_{\alpha}$ \\ \hline
Барабан, $^0$               & 1164         & 1801        & 2801         \\ \hline
$\Delta$ Барабан, $^0$      & 1            & 1           & 1            \\ \hline
$\lambda_{th}$, нм          & 434,05       & 486,13      & 656,28       \\ \hline
$\lambda$, нм               & 434          & 484         & 658          \\ \hline
$\Delta\lambda$, нм         & 0,5          & 0,5         & 0,5          \\ \hline
$\varepsilon(\lambda)$, нм  & 0,000115     & 0,004382    & 0,002621     \\ \hline
$\frac{1}{4}-\frac{1}{n^2}$ & 0,210        & 0,188       & 0,139        \\ \hline
$R_H$, $10^6$ м    $^{-1}$         & 10,972       & 11,02       & 10,94        \\ \hline
$\Delta R_H$, $10^6$ м $^{-1}$     & 0,012        & 0,05        & 0,03         \\ \hline
$H_{av}$, $10^6$ м$^{-1}$          & \multicolumn{3}{c|}{10,98}                 \\ \hline
$\Delta H_{av}$, $10^6$ м $^{-1}$  & \multicolumn{3}{c|}{0,03}                  \\ \hline
\end{tabular}
\caption{итоговые данные измерения $R_H$}
\end{center}
\end{table}
\newpage
В итоге мы получаем, что 
\[R = (10,98 \pm 0,03) \cdot 10^6 \text{м}^{-1}\]
\[R_{th} = 10,9737 \cdot 10^6 \text{м}^{-1}\]

В итоге получаем, что измеренная $R$ с точностью до ошибки равна теоретической.
\newpage
Проведем теперь измерения спектра йода. Определим деления барабана, отвечающие $h\nu_{1,0}, h\nu_{1,5}, h\nu_{\text{гр}}$. Запишем их в таблицу. Далее вычислим различные энергии возбужденного йода и энергии диссоциации молекул в основном и возбужденном состоянии, все запишем в таблицу, как и промежуточные данные.

Так же нам понадобятся некоторые теоретические факты, такие как

Для возбужденного кванта
\[h\nu_2 = (h\nu_{1,5}-h\nu_{1,0})/5\]

и для основного состояния
\[h\nu_1 = 0,027 eV\]

А энергия возбуждения равна $E_A = 0,94 eV$.

Так же мы знаем, что $h\nu_{\text{гр}} = h\nu_e + D_2 = D_1 + E_a$ и $h\nu_{0,n_2} = h_e + h\nu_2\left(n_2 + \frac{1}{2}\right) - 1/2\nu_1$

\begin{table}[h]
\begin{center}
\begin{tabular}{|c|c|c|c|}
\hline
                    & 1,0          & 1,5         & гр          \\ \hline
Барабан, $^0$       & 2643         & 2557        & 2011        \\ \hline
$\lambda$, нм       & 618          & 591         & 509         \\ \hline
$\Delta\lambda$, нм & 1            & 1           & 1           \\ \hline
$h\nu$, $eV$        & 2,007        & 2,099       & 2,437       \\ \hline
$\Delta h\nu$, $eV$ & 0,003        & 0,004       & 0,005       \\ \hline
$h\nu_2$, $eV$      & \multicolumn{3}{c|}{$0,0183 \pm 0,0001$} \\ \hline
$h\nu_{el}$, $eV$   & \multicolumn{3}{c|}{$1,99\pm 0,01$}      \\ \hline
$D_1$, $eV$         & \multicolumn{3}{c|}{$1,50 \pm 0,01$}     \\ \hline
$D_2$, $eV$         & \multicolumn{3}{c|}{$0,45\pm0,01$}       \\ \hline
\end{tabular}
\caption{Итоговая таблица для спектра йода}
\end{center}
\end{table}
\section*{Вывод}
Мы получили спектральные линии водорода, по которым смогли измерить постоянную Ридберга и убедились в теоретическом значении этой константы. Так же мы измерили некоторые энергии для возбуждения атома йода, которые довольно точны.

\end{document}

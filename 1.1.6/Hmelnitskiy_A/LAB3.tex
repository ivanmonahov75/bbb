\documentclass[a4paper, 12pt]{article}

\usepackage[warn]{mathtext}
\usepackage{cmap}	
\usepackage[T2A]{fontenc}			
\usepackage[utf8]{inputenc}		
\usepackage[english,russian]{babel}

\usepackage{caption} 

\usepackage{amsmath,amsfonts,amssymb,amsthm,mathtools} % AMS

\author{Богданов Александр Б05-001}
\title{Работа 1.1.6}
\date{08.10.20}

\begin{document}
	
\maketitle

\begin{center}
\textbf{\Large Изучение электронного осциллографа}
\end{center}

\textbf{Цель работы:}
ознакомление с устройством и работой осциллографа и изучение его основных характеристик.

\textbf{Оборудование:}
осциллограф, генераторы электрических сигналов, соединительные кабели.

\textbf{1.Подготовка к работе.}

\textbf{2.Наблюдение периодического сигнала от генератора и измерение его частоты.}

\begin{center}
	\begin{tabular}{*{7}{| c} |}
		\hline
		$f_{зг}$, Гц & T, дел & с/дел, $10^{-3}$ c & T, $10^{-3}$ c & f, Гц & $\delta$f, Гц & f -$f_{зг}$, Гц\\
		\hline
		1000 & 1	& 1 & 1 & 1000 & 20 & 0 \\
		\hline
		 430 & 1,2 & 2 & 2,4 & 417 &  7 & 10 \\
		\hline
		2160 & 0,9& 0,5 & 0,45 & 2222 & 48 & 40\\
		\hline		
		4600 & 1,1 & 0,2 & 0,22 & 4545 & 83 & 160\\
		\hline
	\end{tabular}
\end{center}
\begin{center}
$\delta f=\varepsilon f \times f = \varepsilon T \times \frac{1}{T}  = \frac{\delta T}{T^2}\text{, где }\delta T = \frac{\text{цена деления}}{10} $
\end{center}

\textbf{3.Измерение амплитуды сигнала.}

$\\$
$f=1 кГц$

$\\$
$U_{min} = 6 мВ$

$\\$
$\delta U_{min} = \frac{5 \times 10^{-3} В}{2} = 2,5 \times 10^{-3}В \approx  3 мВ$

$\\$
$\varepsilon U_{min} = \frac{3мВ}{6мВ} = 50\%$

$\\$
$U_{max} = 10,5 В$

$\\$
$\delta U_{max} = \frac{5 В}{2} = 2,5 В$

$\\$
$\varepsilon U_{max} = \frac{2,5В}{10,5В} = 24\%$

$\\$
Т.е. $U_{min} = (6 \pm 3 ) мВ, U_{max} = (10,5 \pm 2,5) В$

$\\$
$\beta = 10\lg\frac{P_2}{P_1} = 20\lg\frac{U_{max}}{U_{min}} = 20\lg\frac{10,5В}{6\times 10^{-3}В}\approx 64,9дб$

$\\$
\textbf{4.Измерение амплитудно-частотной характеристики осциллографа.}

\[k(f)=\frac{u(f)}{U_0}; 2u_0=20В, \frac{В}{дел} = 5\]
\begin{center}
	\begin{tabular}{*{6}{| c} |}
		\hline
		f, Гц & $\lg f$ & 2$U_{AC}$, дел & $K_{AC} $ &2$U_{DC}$, дел & $K_{DC}$\\
		\hline
		2 & 0,3	& 2 & 0,5 & 4 & 1  \\
		\hline
		5 & 0,7 & 3,2 & 0,8 & 4 &  1 \\
		\hline
		7 & 0,85 & 3,6 & 0,9 & 4 & 1 \\
		\hline		
		9 & 0,95 & 3,6 & 0,9 & 4 & 1 \\
		\hline
		10 & 1 & 3,8 & 0,95 & 4 & 1 \\
		\hline
		$10^2$ & 2 & 4 & 1 & 4 & 1 \\
		\hline
		$10^3$ & 3 & 4 & 1 & 4 & 1 \\
		\hline
		$10^5$ & 5 & 4 & 1 & 4 & 1 \\
		\hline
		$10^6$ & 6 & 4 & 1 & 4 & 1 \\
		\hline
		$2\times 10^6$ & 6,3  & 4 & 1 & 4 & 1 \\
		\hline
		$3\times10^6$& 6,5 & 4 & 1 & 4 & 1 \\
		\hline
		$6\times10^6$ & 6,8 & 4 & 1 & 4 & 1 \\
		\hline
	\end{tabular}
\end{center}



\begin{figure}[h!]
\centering	
\includegraphics[scale=0.1] {KK.jpg}
\end{figure}
 
Различие возникает из-за того, что при закрытом входе у конденстора присутсвует реактивное сопротивление, которое зависит от частоты. 

\textbf{5.Изучение влияния АЧХ на искажение сигнала.}

\begin{figure}[h!]
	\centering	
	\includegraphics[scale=0.1] {BB.jpg}
\end{figure}
При низких частотах при открытом входе меандр сглаживается и становится похож на синусоиду из-за того, что растет реактивное сопротивление конденсатора. При больших частотах искажений не наблюдалось.
$\\$


\textbf{6.Измерение разности фазово-частотных характеристик каналов осциллографа.}

\begin{figure}[h!]
	\centering	
	\includegraphics[scale=0.1] {FF.jpg}
\end{figure}

\[x(t)=A_x \sin (\omega t + \varphi_x); y(t)=A_y \sin (\omega t + \varphi_y)\]
\[\text{Положим }  \omega t = - \varphi_x, \text{тогда } \sin |\varDelta| = |\frac{y_0}{A_y}|\]
\[|\varDelta \varphi| = \arcsin |\frac{y_0}{A_y}|, \text{если элипс направлен вправо}\]
\[|\varDelta \varphi| = \pi - \arcsin |\frac{y_0}{A_y}|, \text{если элипс направлен влево}\]
\begin{center}
	\begin{tabular}{*{6}{| c} |}
		\hline
		f, $10^6$Гц & $\lg f$ & $|2y_0|$, дел & $|2A_y|$, дел & $\arcsin |\frac{y_0}{A_y}|$, рад & $|\varDelta \varphi|, рад $\\
		\hline
		0,2 & 5,3 & 0,4 & 4 & 0,1 & 0,1 \\
		\hline
		0,4 & 5,6 & 0,8 & 4 & 0,2 & 0,2 \\
		\hline
		0,6 & 5,8 & 1,2 & 4 & 0,3 & 0,3 \\
		\hline		
		0,8 & 5,9 & 1,6 & 4 & 0,4 & 0,4 \\
		\hline
		1,0 & 6,0 & 2,0 & 4 & 0,5& 0,5 \\
		\hline
		1,2 & 6,1 & 2,4 & 4 & 0,6 & 0,6 \\
		\hline
		1,4 & 6,1 & 2,8 & 4 & 0,8 & 0,8 \\
		\hline
		1,6 & 6,2 & 3,2 & 4 & 0,9 & 0,9 \\
		\hline
		1,8 & 6,3 & 3,5 & 4 & 1,1 & 1,1 \\
		\hline
		2,0 & 6,3 & 3,6 & 4 & 1,1 & 1,1 \\
		\hline
		2,2 & 6,3 & 3,8 & 4 & 1,3 & 1,3 \\
		\hline
		2,4 & 6,4 & 4,0 & 4 & 1,6 & 1,6 \\
		\hline
		2,6 & 6,4 & 4,0 & 4 & 1,6 & 1,6 \\
		\hline
		2,8 & 6,4 & 4,0 & 4 & 1,6 & 1,6 \\
		\hline
		3,0 & 6,5 & 3,8 & 4 & 1,3 & 1,8 \\
		\hline
		3,2 & 6,5 & 3,6 & 4 & 1,1 &  2,0 \\
		\hline
		3,4 & 6,5 & 3,2 & 4 & 0,9 & 2,2\\
		\hline
		3,6 & 6,6 & 2,8 & 4 & 0,8 & 2,3 \\
		\hline
		3,8 & 6,6 & 2,4 & 4 & 0,6 & 2,5 \\
		\hline
		4,0 & 6,6 & 2,0 & 4 & 0,5 & 2,6 \\
		\hline
		4,2 & 6,6 & 1,8 & 4 & 0,5 & 2,6 \\
		\hline
		4,4 & 6,6 & 1,6 & 4 & 0,4 & 2,7 \\
		\hline
		4,6 & 6,7 & 1,2 & 4 & 0,3 & 2,8 \\
		\hline
		4,8 & 6,7 & 0,8 & 4 & 0,2 & 2,9 \\
		\hline
		5,0 & 6,7 & 0,6 & 4 & 0,2 & 2,9 \\
		\hline
		5,2 & 6,7 & 0,4 & 4 & 0,1 & 3,0 \\
		\hline		
	\end{tabular}
\end{center}

\begin{figure}[h!]
	\centering	
	\includegraphics[scale=0.15] {QQ.jpg}
\end{figure}

$\\$
$\\$
$\\$
$\\$
$\\$
$\\$

\textbf{7. Наблюдение фигур Лиссажу и измерение частоты.}

\begin{figure}[h!]
	\centering	
	\includegraphics[scale=0.1] {PP.jpg}
\end{figure}

Проведем вертикальную и горизонтальную прямые так, чтобы они перескали фигуру не в узлах. Отношение количества пересечений по разным осям и есть отношение частот сигналов.


\textbf{8. Измерение АЧХ интегрирующей и дифференцирующей RC-цепочек}

\begin{figure}[h!]
	\centering	
	\includegraphics[scale=0.1] {MM.jpg}
\end{figure}

\[a)K_a= \frac{1}{\sqrt{(\omega t)^2 + 1}} = \frac{1}{\sqrt{(2\pi f RC)^2 + 1}}\]

\begin{center}
	\begin{tabular}{*{4}{| c} |}
		\hline
		f, кГц & Вход, дел &  Выход, дел & $K_a$ \\
		\hline
		0,83 & 4 & 4,0 & 1,0  \\
		\hline
		2,4 & 4 & 3,6 & 0,9 \\
		\hline
		3,9 & 4 & 3,2 & 0,8 \\
		\hline		
		5,3 & 4 & 2,8 & 0,7  \\
		\hline
		7,3 & 4 & 2,4 & 0,6   \\
		\hline
		10 & 4 & 2,0 & 0,5  \\
		\hline
		12,5 & 4 & 1,6 & 0,4  \\
		\hline
		17,1& 4 & 1,2 & 0,3  \\
		\hline
		27 & 4 & 0,8 & 0,2 \\
		\hline
		52,7 & 4 & 0,4 & 0,1 \\
        \hline
        285 & 4 & 0,2 & 0,05 \\
        \hline
	\end{tabular}
\end{center}

\[a)K_b= \frac{1}{\sqrt{(\omega t)^{-2} + 1}} = \frac{1}{\sqrt{\frac{1}{(2\pi f RC)^2} + 1}}\]

\begin{center}
	\begin{tabular}{*{4}{| c} |}
		\hline
		f, кГц & Вход, дел &  Выход, дел & $K_b$ \\
		\hline
		1600 & 4 & 4,0 & 1,0  \\
		\hline
		16,7 & 4 & 3,6 & 0,9 \\
		\hline
		8,5 & 4 & 3,2 & 0,8 \\
		\hline		
		5,7 & 4 & 2,8 & 0,7  \\
		\hline
		4,4 & 4 & 2,4 & 0,6   \\
		\hline
		3,3 & 4 & 2,0 & 0,5  \\
		\hline
		2,3 & 4 & 1,6 & 0,4  \\
		\hline
		1,7  & 4 & 1,2 & 0,3  \\
		\hline
		1 & 4 & 0,8 & 0,2 \\
		\hline
		0,4 & 4 & 0,4 & 0,1 \\
		\hline
		0,2 & 4 & 0,3 & 0,08 \\
		\hline
	\end{tabular}
\end{center}

\begin{figure}[h!]
	\centering	
	\includegraphics[scale=0.15] {M1.jpg}
\end{figure}

\begin{figure}[h!]
	\centering	
	\includegraphics[scale=0.15] {M2.jpg}
\end{figure}

\begin{figure}[h!]
	\centering	
	\includegraphics[scale=0.15] {M4.jpg}
\end{figure}

\begin{figure}[h!]
	\centering	
	\includegraphics[scale=0.15] {M3.jpg}
\end{figure}
\centering
Записи с работы и подпись
\end{document}
\documentclass[a4paper, 10pt]{article}%тип документа

%Русский язык
\usepackage[T2A]{fontenc} %кодировка
\usepackage[utf8]{inputenc} %кодировка исходного кода
\usepackage[english,russian]{babel} %локализация и переносы

%Вставка картинок
\usepackage{graphicx}
\graphicspath{{pictures/}}
\DeclareGraphicsExtensions{.pdf,.png,.jpg}

%Графики
\usepackage{pgfplots}
\pgfplotsset{compat=1.9}

%Математика
\usepackage{amsmath, amsfonts, amssymb, amsthm, mathtools}

%Заголовок
\author{Валеев Рауф Раушанович \\
группа 825}
\title{Работа 1.1.1 \\
Определение систематических и случайных погрешностей при измерении удельного сопротивления нихромой проволоки}
\begin{document}
\maketitle 
\newpage
В работе используеются: линейка, линейка, штангенциркуль, микрометр, отрезок проволоки из нихрома, амперметр, вольтметр, источник ЭДС, мост постоянного тока, реостат, ключ. 
\begin{enumerate}
\item Точность измерения с помощью штангенциркуля -- 0,1 мм. Точность измерения с помощью микрометра -- 0,01 мм.
\item Измеряем диаметр проволки с помощью штангенциркуля ($d_1$, табл. 1) и микрометра ($d_2$, табл. 2) на 10 различных участках.

При измерении диаметра проволоки штангенциркулем случайная погрешность отсутствует. Следовательно, точность резульата определяется только точностью штангенциркуля $\Rightarrow  d_1 = (0,4 \pm 0,1)$ мм 

При измерении микрометром есть как систематичская, так и случайная ошибка:
\[ \sigma_{\text{сист}}  = 0,01 \text{ мм, }
\sigma_{\text{сл}}  = \dfrac{1}{N} \cdot \sqrt{\sum_{i = 1}^N (d - \overline{d})^2} = \frac{1}{10}\sqrt{2,4\cdot10^{-4}} \approx 1,6\cdot10^{-3} \text{ мм}\]
\[ \sigma =  \sqrt{\sigma_{\text{сист}}^2 + \sigma_{\text{сл}}^2} \approx 0,01 \text{мм}\]
\[ d_2 = (0,364 \pm 0,1) \text{мм}\]
Поскольку погрешность микрометра на порядок меньше погрешности штангенциркуля, для расчета площади поперечного сечения проволоки будем использовать значение, полученное измерением с помощью микрометра $ \Rightarrow $
\item Определим площадь поперечного сечения проволоки: 
\[ S = \dfrac{ \pi d^2 }{4} = \dfrac{3,14 \cdot (0,364)^2}{4} \approx 0,104 \text{мм}\]
Погрешность находим по формуле:
\[ \left( \dfrac{\sigma_{s}}{S} \right)^2 =\dfrac{2 \text{ }\sigma_{d_2}}{d_2} \Rightarrow \dfrac{\sigma_{s}}{S} = \dfrac{\sqrt{2} \text{ } \sigma_{d_2}}{d_2} \Rightarrow \sigma_{s} =\dfrac{\sqrt{2} \text{ } \sigma_{d_2} S}{d_2} \approx 4,04 \cdot 10^{-3} \text{мм}\]
\item см. табл. 2
\item Очевидно, что надо мерять способом показанным на рис. 1a, так как:

для схемы на рисунке 1а: $R_{\text{пр}}$/$R_{\text{V}}$ = 5/400 = 0,0125, т.е. 1,25%;

а для схемы на рисунке 1б: $R_{\text{A}}$/$R_{\text{пр}}$ = 1,2/5 = 0,24, т.е. 24%.
\newpage
\item Собираем схему рис. 1 \\
\begin{figure}[h]
\center{\includegraphics{111.png}}
\end{figure}
\item Опыт проводим для трех величин: 
$ l_1 = (50\pm 0,1) \text{ см}, l_2 = (30\pm0,1) \text{ см}, l_3 = (20\pm 0,1) \text{ см.}$ \\
Измерения ведем для возрастающих и убывающих значений тока, все измерения записываем в табл. 3, табл. 4, табл. 5.
\item Строим графики зависимостей V = f(I) для всех трех отрезков проволоки, так как 1 прямая не проходит через все точки, но с точность до погрешностей мы ее провести можем, то, ищем график прямой V = f(I) по формуле
\[ V = \dfrac{ \left\langle VI \right\rangle }{ \left\langle I^2 \right\rangle } x \]
\begin{tikzpicture}
\begin{axis} [title = $l_1$,
		xlabel=I,
		ylabel=V]
\addplot coordinates {
( 464, 90 )
( 436, 84.2 )
( 420, 81.35 )
( 412, 80.14 )
( 356, 69.1 )
(260, 51.17)
(180, 34.16)
(152, 29.44)
(80, 14.91)
(0, 0)
};
\end{axis}

\begin{axis}
\addplot [red] coordinates { (0,0) (464, 90) };
\end{axis}
\end{tikzpicture} \\
\begin{tikzpicture}
\begin{axis} [title = $l_2$,
		xlabel=I,
		ylabel=V]
\addplot coordinates {
( 288, 93.17 )
( 252, 80.78 )
( 236, 74.32 )
( 176, 55.79 )
( 146, 46.68 )
( 104, 46.68 )
( 92, 29.69 )
( 68, 22 )
( 24, 7.55 )
(0, 0)
};
\end {axis}

\begin{axis}
\addplot [red] coordinates { (0,0) (288, 92.13) };
\end{axis}
\end{tikzpicture} \\
\begin{tikzpicture}
\begin{axis} [title = $l_3$,
		xlabel=I,
		ylabel=V]
\addplot coordinates {
( 204, 96.46 )
( 184, 87.23 )
( 152, 71.6 )
( 128, 61.39 )
( 100, 47.37 )
( 68, 33.56 )
( 60, 27.1 )
( 28, 11.92 )
( 0, 0 )
};
\end {axis} 
\begin{axis}
\addplot [red] coordinates { (0,0) (204, 96.68) };
\end{axis}
\end{tikzpicture}
\item Запишем в табл. 6 данные средних значений некоторых величин, которые мы в дальнейшем будем использовать.
\item По формулам 
\[ \sigma^{\text{случ}}_{R_{\text{ср}}} = \dfrac{1}{\sqrt{N}}\cdot \sqrt{\dfrac{\left\langle V^2 \right\rangle }{\left\langle I^2 \right\rangle} - R^{2}_{\text{ср}}}\] 
\[ \sigma^{\text{сист}}_{R_{\text{ср}}} = R_{\text{ср}}\sqrt{\left( \dfrac{\sigma_{V}}{V} \right) ^{2} + \left( \dfrac{\sigma_{I}}{I} \right) ^{2} } \]
\[ \sigma_R =  \sqrt{\sigma_{\text{сист}}^2 + \sigma_{\text{сл}}^2} \]
\[ R_\text{ср} = \dfrac{\left\langle VI \right\rangle}{\left\langle I^2 \right\rangle}\]
Находим сопротивления и погрешности для каждого из участков проволоки. Данные заносим в табл.7. В эту же таблицу заносим результаты измерения сопротивления мостом Уитстона (P4833), изображенном ниже.
\begin{figure}[h]
\center{\includegraphics{1111.png}}
\end{figure}
\item по формулам 
\[ \sigma_{\rho} = \rho \sqrt{\left( \dfrac{\sigma_{R}}{R} \right) ^{2} + \left( \dfrac{\sigma_{l}}{l} \right) ^{2} + \left( \dfrac{\sigma_{S}}{S} \right) ^{2} } \]
\[\rho = \dfrac{R \cdot S}{l}\]
находим удельное сопротивление и погрешность для каждой из длин проволоки и заносим эти значения в табл.8.
\end{enumerate}
Окончательно:$\rho = (1,08\pm0,04) \dfrac{\text{Ом} \cdot \text{мм}^2}{\text{м}}$

Полученное значение удельного сопротивления сравниваем с табличными значениями. В справочнике (Физические велечины. М.: Энергоиздат, 1991. С. 444) для удельного сопротивления нихрома при 20 $ ^\circ C$ в зависимости от массового содержания компонента сплава меняются в промежутке (1,12--0,97) $\dfrac{\text{Ом} \cdot \text{мм}^2}{\text{м}}$. Полученное значение наиболее близко к значению 1,06 $\dfrac{\text{Ом} \cdot \text{мм}^2}{\text{м}}$ для сплава с содержанием 78 процентов Никеля, 20 процентов Хрома и 2 Марганца (проценты по массе).
\newpage
\begin{table}
	\caption{Результаты измерения диаметра проволоки}
	\begin{tabular}{|r|c|c|c|c|c|c|c|c|c|c|c|}
	\hline
& 1 & 2 & 3 & 4 & 5 & 6 & 7 & 8 & 9 & 10\\
\hline
$d_1$, мм & 0,4 & 0,4 & 0,4 & 0,4 & 0,4 & 0,4 & 0,4 & 0,4 & 0,4 & 0,4 \\
\hline
$d_2$, мм & 0,37 & 0,36 & 0,36 & 0,36 & 0,37 & 0,36 & 0,37 & 0,37 & 0,36 & 0,36 \\
\hline
\multicolumn{1}{|r|}{} & \multicolumn{5}{c}{ \( \overline{d_{1}} = 0,4 мм \) мм}  & \multicolumn{5}{c|}{ \( \overline{d_{2}} = 0,364 мм \) мм}\\
\hline
\end{tabular}
\end{table}

\begin{table}
\caption{Основные характеристики амперметра и вольтметра}
\begin{tabular}{|p{3cm}|p{5cm}|p{5cm}|}
\hline
& Вольтметр & Амперметр \\
\hline
Система & Магнитоэлектрическая & Электромагнитная\\
\hline
Погрешность& Класс точности: 0,5& 0,002X + 2k, где X - значение измеряемой велечини, а k - единица младшего разряда\\
\hline
Предел измерений $ x_{n} $& 0,6 В & автоматически настраивается в зависимости от силы тока \\
\hline
Число делений шкалы $n$& 150 & -- \\
\hline
Цена делений $x_n / n$& 4 мВ/дел & -- \\
\hline
Чувствительность $ n / x_n$& 250 дел/В & --\\
\hline
Абсолютная погрешность $ \vartriangle x_M$& 1,5&--\\
\hline
Внутреннее сопротивление прибора (на данном пределе измерений)& 400 Ом & 1,2 Ом\\
\hline
\end{tabular}
\end{table}
\begin{table}
\caption{Результаты ВАХ для $l_1$}
\begin{tabular}{|r|c|c|c|c|c|c|c|c|c|c|}
\hline
&1&2&3&4&5&6&7&8&9&10 \\
\hline
V, мВ&464&420&356&180&80&0&152&260&412&436 \\
\hline
$\sigma_V$, мВ& 0,93&0,84&0,71&0,36&0,16&0&0,3&0,52&0,82&0,87 \\
\hline
 I, мА&90&81,35&69,1&34,16&14,91&0&29,44&51,17&80,14&84,2 \\
 \hline
 $\sigma_I$, мA&0,2&0,18&0,16&0,09&0,05&0,0002&0,08&0,12&0,18&0,19 \\
 \hline
\end{tabular}
\caption{Результаты ВАХ для $l_2$}
\begin{tabular}{|r|c|c|c|c|c|c|c|c|c|c|}
\hline
&1&2&3&4&5&6&7&8&9&10 \\
\hline
V, мВ&288&236&146&92&24&0&68&104&176&252 \\
\hline
$\sigma_V$, мВ&0,58&0,47&0,3&0,18&0,05&0&0,14&0,21&0,35&0,5 \\
\hline
 I, мА&93,17&74,32&46,68&29,69&7,55&0&22&34,52&55,79&80,78 \\
 \hline
 $\sigma_I$, мA&0,21&0,17&0,11&0,06&0,02&0,0002&0,05&0,07&0,13&0,18 \\
 \hline
\end{tabular}
\caption{Результаты ВАХ для $l_1$}
\begin{tabular}{|r|c|c|c|c|c|c|c|c|c|}
\hline
&1&2&3&4&5&6&7&8&9 \\
\hline
V, мВ&204&152&100&60&0&28&68&128&184 \\
\hline
$\sigma_V$, мВ&0,41&0,3&0,2&0,12&0&0,06&0,14&0,26&0,37 \\
\hline
 I, мА&96,46&71,6&47,37&27,1&0&11,92&33,56&61,39&87,23 \\
 \hline
 $\sigma_I$, мA&0,21&0,14&0,11&0,07&0,0002&0,04&0,08&0,14&0,19 \\
 \hline
\end{tabular}
\end{table}
\newpage
\begin{table}
\caption{Средние величины}
\begin{tabular}{|r|c|c|c|c|c|}
\hline
&$\left\langle V \right\rangle$ & $\left\langle I \right\rangle$ & $\left\langle I^2 \right\rangle$ & $\left\langle  V^2 \right\rangle$  & $\left\langle IV \right\rangle$ \\
\hline
$l_1$&276&53,45&100777,6&3783,9&19537,6 \\
\hline
$l_2$&138,6&44,45&27891,6&2863,52&8936,21 \\
\hline
$l_3$&92,4&43,66&13396,8&3005,56&6344,81 \\
\hline
\end{tabular}
\end{table}
\begin{table}
\caption{Результаты измерения сопротивления провлоки}
\begin{tabular}{|c|c|c|}
\hline
$l_1$&$l_2$&$l_3$ \\
\hline
$R_0$ = 5,1472 Ом (по Р4833) & $R_0$ = 3,0999 Ом (по Р4833) & $R_0$ = 2,0857 Ом (по Р4833) \\
$R_\text{ср}$ = 5,16 Ом  & $R_\text{ср}$ = 3,12 Ом  & $R_\text{ср}$ = 2,11 Ом  \\
$\sigma^{\text{случ}}_{R}$ = 0,07 Ом&$\sigma^{\text{случ}}_{R}$ = 0,008 Ом&  $\sigma^{\text{случ}}_{R}$ = 0,008 Ом \\
$\sigma^{\text{сист}}_{R}$ = 0,015 Ом&$\sigma^{\text{сист}}_{R}$ = 0,014 Ом&  $\sigma^{\text{сист}}_{R}$ = 0,012 Ом \\
$\sigma_{R}$ = 0,07 Ом&$\sigma_{R}$ =  0.017 Ом&$\sigma_{R}$ =   0,014 Ом \\
\hline
\end{tabular}
\end{table}
\begin{table}
\begin{tabular}{|c|c|c|}
\hline
l, м& $\rho$, $\dfrac{\text{Ом} \cdot \text{мм}^2}{\text{м}}$& $\sigma_{\rho}$, $\dfrac{\text{Ом} \cdot \text{мм}^2}{\text{м}}$ \\
\hline
0,5&1,07&0,04 \\
0,3&1,08&0,04 \\
0,2&1,09&0,04 \\
\hline
\end{tabular}
\end{table}
\end{document}